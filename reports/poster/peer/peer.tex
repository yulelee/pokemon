\documentclass[12pt]{article}
\usepackage{fullpage, enumitem, amsmath, amssymb, 
	graphicx, dirtytalk, tikz, listings, titlesec}

\usepackage[margin=0.8in]{geometry}

\makeatletter
\@addtoreset{equation}{enumi}
\makeatother

\graphicspath{ {images/} }
\titlespacing*{\section}{0pt}{0.5ex plus 1ex minus .2ex}{0.5ex plus .2ex}

\graphicspath{ {images/} }

\begin{document}

\begin{center}
{\Large CS221 Project Poster Peer Review}

\begin{tabular}{rl}
\newline
\\ 
SUNet ID: & yulelee \\
Name: & Yue Li \\
\end{tabular}
\end{center}

\begin{enumerate}[label=(\roman*)]
    \item \textbf{AI Agent for Chinese Chess}
    \par
    \textit{Li Deng}
    \par
    This project uses various strategies (minimax, alpha-beta pruning, Monte Carlo tree search, etc) to let the AI agent play Chinese chess. I like this project because at the presentation, he has a live demo with nice GUI so that I can play the chess with his AI agent. (It was really fun and the agent is actually quite impressive!) I have to say that this is my favorite project today!
    \item 
    \textbf{Application of AI in Intrusion Sensing Security System}
    \par
    \textit{Quan Yang, Xin Xie, Ruming Zhen}
    \par
    This project uses the signal from a intrusion sensor and try to determine whether
    an intrusion truly happened. The team formulates the signals as an HMM, and the 
    classification result is really great. (Although this might be more like a machine learning project, but their algorithm was pretty cool).
    
    \item
    \textbf{AI Agent for Atari Space-Invader Game}
    \par
    \textit{Zhiwen Zhang, Yuanlin Wen, Yiming Chen}
    \par
    This project utilize deep Q-networks, and use the image of the game directly 
    as the input, to train an AI agent to play space-invader. I'm interest in this
    project because this topic is covered in one of the CS221 sections, 
    and I was really 
    curious about it at that time.
    
    
\end{enumerate}
\end{document} 










